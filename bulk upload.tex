\section{BETYdb: Bulk Data Upload}

\subsection{Overview}
 
There are three phases for a basic bulk upload of data: 

\begin{enumerate}

\item Use the web interface

      \begin{itemize}
    \item to enter metadata pertaining to your data set (new sites, species, cultivars, citations, or treatments);
    \item to obtain a template appropriate for your data set.
    \end{itemize}
\item Fill in the template with your data. There are four templates to choose from:
\begin{itemize}
\item \href{https://docs.google.com/spreadsheets/d/1maK1uKr6i9KERaYdU5zSiXcBndQoiG4Vgn2DTnqdfbA/export?format=csv&gid=0}{yields.csv} --- Use this template if you are uploading yield data and you wish to specify the citation in the file by author, year, and title.

If your data includes standard error and cultivar information and you do not plan to specify any of the required information interactively, you will be able to use this template "as-is".  Otherwise, you will need to delete one or more columns:
\begin{enumerate}
\item If your data has no standard error information, delete both the \verb|SE| and the \verb|n| column.
\item If your data set has a single uniform value for the site, species, cultivar, treatment, access\_level, or date, then these values may be entered interactively through the web interface; in this case you should delete the corresponding column(s) from the template. \item Note that cultivar information can't be specified interactively unless species information is as well; delete the \verb|cultivar| column if and only if you either have no cultivar information or you are specifying both the species and the cultivar interactively.
\end{enumerate}
\item \href{https://docs.google.com/spreadsheets/d/1ExLosMvX05jHWO9UYVE4Dxcl2ZbUgPc0KYoUPruaOtM/export?format=csv&gid=0}{yields\_by\_doi.csv} --- Use this template if you are uploading yield data and you wish to specify the citation in the file by doi.

Again, if you do not have data for all of the columns listed in the template, or if you plan to specify some of the data interactively, you will have to delete one or more columns.

You may also use this template if all of the data in your data set pertains to a single citation and you wish to specify that citation interactively.  In this case, you must delete the \verb|citation_doi| column.

\item \href{https://docs.google.com/spreadsheets/d/1Bv4dAPKU6YDJ6yB0DC4bAmHoGxSLgKybMpTR7qBvCu0/export?format=csv&gid=0}{traits.csv} --- Use this template if you are uploading trait data and you wish to specify the citation in the file by author, year, and title.

\textbf{This template must be modified before it can be used.}  In particular, the column headings \verb|[trait variable 1]| \dots \verb|[trait variable n]| must be replaced by actual variable names that \textit{exactly} match names of variables in the database that have been marked to be recognized as trait variables.  The number of these trait variable columns may need to be increased or decreased to accomodate the data set.

Some trait variables allow or even require corresponding covariate information to be included.  Again, the column headings \verb|[covariate 1]| \dots \verb|[covariate n]| must be changed to actual covariate variable names, and the number of these columns may need to be increased or decreased to match the available information.  As with the yield data templates, some columns may also need to be deleted.  For a list of recognized trait variable names and their corresponding required and optional covariates, visit the trait variable/covariates list at www.betydb.org.  [TO-DO: Make this Web page.]
\item \href{https://docs.google.com/spreadsheets/d/1Bv4dAPKU6YDJ6yB0DC4bAmHoGxSLgKybMpTR7qBvCu0/export?format=csv&gid=0}{traits\_by\_doi.csv} --- As with the corresponding yield data template, use this template if you are uploading trait data and you wish to specify the citation in the file by doi or if you plan to specify the citation interactively (in which case delete the \verb|citation_doi| column).  \textbf{Again, this template must be modified before it can be used.}
 \end{itemize}
\item Use the web interface to upload your data set and insert it into the database.
\end{enumerate}


\textit{In what follows, the term "field" always refers either to a column name used in the heading of the uploaded CSV file or to an entry in that column in some particular row of the file.  On the other hand, and the term "column" may either refer to a column of data in the uploaded CSV file or to an attribute of a trait or yield datum in the traits or yields table of the database.}

\subsection{Detailed CSV Data File Specifications}

\subsubsection{Required fields}

\begin{enumerate}
\item For yields uploads, the only required field is a \verb|yield| column.
\item For trait uploads, there must be at least one column whose label exactly matches the variable name for the trait value being specified.  (Leading and trailing spaces are permitted, but letter case must exactly match the name of the variable specified in the database.)  If this trait variable has any required covariates, columns for these covariates must be included.
\end{enumerate}

\subsubsection{Information that is required but that \textit{may} be specified interactively for the entire dataset.}

\textit{Data values may be specified interactively only if there is a single value that pertains to the whole data set.}

\textbf{\textit{Information that references existing database entries}}

\begin{enumerate}

\item Citation

\begin{itemize}
\item If only one citation for the entire dataset exists, it may be specified
  interactively by choosing a citation on the citations page instead of
  including citation information in the CSV file.
\item Otherwise, specify the citation in the CSV file, either by doi or by
  author, year, and title.
\item If a DOI is available for all citations in the data set, the citation
  corresponding to each row may be specified in a \verb|citation_doi| column.  In
  this case, the \verb|citation_author|, \verb|citation_year|, and \verb|citation_title|
  columns must not be in the column heading list.  (If such information is
  already included in the data set, to keep such columns for purely
  informational purposes, the string \verb|-ignore| may be appended to each of
  these headings.  One might want to do this, for example, to keep a visual
  record of the author, year, and title even when it is the citation doi
  that is being used to determine how the data will associated with a
  citation in the database.)  Each value in the \verb|citation_doi| column must
  exactly match the \verb|doi| attribute of some row in the \verb|citations| table except
  that letter case and leading and trailing spaces are ignored.
\item Conversely, if a DOI is not available for all citations in the data set,
  or if it is preferred to specify the citation by author, year, and title,
  then the \verb|citation_doi| column should \textit{not} be included and the columns
  \verb|citation_author|, \verb|citation_year|, and \verb|citation_title| must all be
  present.  (Again, if some DOI information is already included and you wish
  to retain it for purely informational purposes, simply give the column
  some descriptive name other than \verb|citation_doi| and it will be ignored by
  the upload code.)
\end{itemize}

\item Site

\begin{itemize}
\item If all of the data in the data set pertains to a single site, that site
  may be specified interactively.
\item Otherwise, a \verb|site| column is required.  The value must match an existing
  \verb|sitename| column value in the \verb|sites| table of the database.  (Letter
  case, leading and trailing spaces, and extra internal spaces are ignored
  when searching for a match.)
\end{itemize}

\item Species

\begin{itemize}
\item If all of the data in the data set pertains to a single species, that
  species may be specified interactively.
\item Otherwise, the \verb|species| column is required.  The value must match an
  existing \verb|scientificname| column value in the \verb|species| table of the
  database.  (Again, letter case, leading and trailing spaces, and extra
  internal spaces are ignored when searching for a match.)
\end{itemize}

\item Treatment

\begin{itemize}
\item If a single treatment and a single citation applies to all of the data in
  the data set, the treatment may be specified interactively provided that
  the citation is specified interactively as well.
\item Otherwise, a \verb|treatment| column is required.  The value must match an
  existing \verb|name| column value in the \verb|treatments| table of the database; moreover, this matching treatment must be consistent with the specified citation.
  (Again, letter case, leading and trailing spaces, and extra internal
  spaces are ignored when searching for a match.)
\end{itemize}

\end{enumerate}

\textbf{\textit{Other information that may be specified interactively}}
    
\begin{enumerate}
\item Date

\begin{itemize}
 \item If a single date applies to all of the data in the data set, the date may be specified interactively.
 \item Otherwise, a \verb|date| column is required.
 \item Date values must be in one of the forms "2003-07-25", "2003-07", or "2003".
\end{itemize}

\item Rounding

\begin{itemize}
 \item The amount of rounding for numerical data can only be specified
   interactively.  Any value from 1 to 4 significant digits may be chosen.
   The amount of rounding for the standard error SE (if present) may be
   specified separately from the amount of rounding for yield and for trait
   variables and their covariates.
 \item By default, all numerical data is rounded to three significant digits.
   For example, with this default in place, 999.1 will be rounded to 999 and
   1001.1 will be rounded to 1000.
\end{itemize}

\end{enumerate}

\subsubsection{Numerical Data  (This is \textit{never} specified interactively.)}

\textbf{\textit{Data for Yields}}

\begin{enumerate}
\item Yield
  
  Every yield data upload file must have a \verb|yield| column.  The data in
   this column must always be a parsable non-negative number and must never
   be blank.  Scientific notation is not currently supported.  As noted
   above, the number given in the file is subject to rounding before being
   inserted into the database.
\item Sample Size
  
 An \verb|n| column is required if and only if an \verb|SE| column is included.  The value must always be an integer greater than 1.
\item Standard Error
  
   An \verb|SE| column is required if and only if an \verb|n| column is included;
    this datum will be inserted into the \verb|stat| column of the \verb|yields| table,
    and the \verb|statname| column value will be set to "SE".
\end{enumerate}

\textbf{\textit{Data for Traits}}

\begin{enumerate}

\item Trait variable values

\begin{itemize}
\item Every trait data upload file must have at least one column whose heading
  matches the name of some recognized trait variable.  A list of
  recognized trait variables is listed on the BetyDB web site.  If
  multiple trait variable columns are used, each row in the CSV file will
  produce one row in the \verb|traits| table for each trait variable column.
  (These resulting rows will be effectively \textit{grouped} by assigning them a
  unique entity id.  Said another way, there is a one-to-one
  correspondence between rows in the CSV file and resultant rows in the
  \verb|entities| table, the table that keeps track of this grouping.)  As with
  yield numbers, the data in this column must always be a parsable number
  and is subject to rounding before being inserted into the database.  In
  addition, it must conform to any range restrictions on the value of the
  variable.
\item The template for traits uploads provides dummy column headings
\verb|[trait variable 1]|, \verb|[trait variable 2]|, etc., which must be changed to
   actual variable names before data can be uploaded.
\end{itemize}

\item Covariate values

\begin{itemize}
\item If any of the included trait variables has a required covariate, there
  must be a column corresponding to that covariate.
\item For any of the included trait variables that has an optional covariate, a
  column corresponding to that covariate \textit{may} be included.
\item The template for traits uploads provides dummy column headings
 \verb|[covariate 1]|, \verb|[covariate 2]|, etc., which must be changed to actual
 variable names before data can be uploaded.
 \end{itemize}
 
\item Sample Size and Standard Error

 To enter \verb|n| and \verb|SE|, add additional columns
 \verb|[trait variable 1] n| and \verb|[trait variable 1] SE|, etc. or
 \verb|[covariate 1] n| and \verb|[covariate 1] SE|, etc. as needed.  Again,
 values of \verb|n| must be at least 2, and columns for \verb|n| and \verb|SE|
 must both be present or both be absent for any particular trait variable or
 covariate.

  \end{enumerate}

\subsubsection{Optional data}

\begin{enumerate}
\item Sample Size and Standard Error

  As noted above, these are both optional, but if one of these is
   included, the other must be included as well.  In other words, the
   column heading list must include both of \verb|n| and \verb|SE| (or, in the case of traits, \verb|[trait or covariate variable k] n| and \verb|[trait or covariate variable k] SE|) or neither.
   Note that if \verb|n| and \verb|SE| are not given fields of the uploaded CSV
   file, the value of the \verb|n| column of the traits or yields table will
   default to 1 and the \verb|stat| and \verb|statname| column values will default
   to NULL.
\item Cultivar

\begin{itemize}
\item If a uniform value for the species is provided interactively when
   uploading the data set, the cultivar may be specified this way as
   well, provided that it also has a uniform value for the whole data
   set.
\item Otherwise, to include cultivar information in the upload file, both a
   \verb|species| and a \verb|cultivar| column must be included.  The values in
   the \verb|cultivar| column are allowed to be blank (in which case a value
   of NULL is inserted into the \verb|cultivar_id| column for the given row);
   but if provided, the value must match the value of the \verb|name| column in
   some row of the \verb|cultivars| table, and moreover, this row must be a
   row associated with the species corresponding to the value given in
   the \verb|species| column.  Again, matching is case insensitive, and
   leading, trailing, and excess internal whitespace is ignored.
\end{itemize}

\item Notes

 To include notes, use a \verb|notes| column.  There is no restriction on
   what can be included in this column, but leading and trailing space
   will be stripped before insertion into the database.  Non-ascii
   characters entered in the file in UTF-8 encoding are allowed.  If
   there is no \verb|notes| column, each row inserted into the \verb|traits| or
   \verb|yields| table will use the empty string as the value for the \verb|notes|
   column.

   \end{enumerate}
